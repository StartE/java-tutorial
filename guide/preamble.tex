%%%%%%%%%%%%%%%%%%
% 标题左对齐
%\usepackage[raggedright]{titlesec}
% 页边距设置
% marginpar=2cm解决margin note在ctex book下无法留出足够空白问题
% 标准的16开,适合于计算机类图书出版
\usepackage[paperheight=260mm, paperwidth=185mm, marginpar=2cm,top=3cm,bottom=3cm,left=2.54cm,right=2.54cm]{geometry}
% 使用A4打印出版,需要根据实际打印纸张尺寸设定
\usepackage[cam,a4,center]{crop}

% A4版本,适合打印
%\usepackage[paperheight=297mm, paperwidth=210mm, marginpar=2cm,top=1.65cm,bottom=1.65cm,left=2cm,right=2cm]{geometry}

% 确定使用了tikz绘制的图片并且在图片中包含beamer中的overlay设置才需要此宏包
%\usepackage{beamerarticle}
% 页眉页脚设置
\usepackage{fancyhdr}
\pagestyle{fancy} % 默认的效果也可以接受
\fancyhf{}                                                  % 清空页眉页脚
\fancyhead[LE,RO]{\thepage}                                 % 页码:偶数页左,奇数页右
\fancyhead[RE]{\leftmark}                                   % 偶数页右
\fancyhead[LO]{\rightmark}                                  % 奇数页左
\fancypagestyle{plain}{ % 重新定义plain样式,在章等的首页使用plain样式 
    \fancyhf{}
    \cfoot{\thepage} % 页脚中间显示页码
    \renewcommand{\headrulewidth}{0pt} % 清除页眉线
}
% 重新设置fancyhdr的headheight,避免报告Package Fancyhdr Warning: \headheight is too small (12.0pt)
\setlength{\headheight}{14pt} 

% 目录样式:解决目录中的省略号太稀疏问题
%\usepackage{tocloft} %暂时不需要这么重量级的宏包
\renewcommand\@dotsep{1} %默认值是4.5

% 图片样式:caption字体小一号,参见caption package guide
\usepackage[margin=10pt,font=small,labelfont=bf,labelsep=endash]{caption}

% 表格样式
% caption在表格上面
% 表格字体小一号
\usepackage{floatrow}
\floatsetup[table]{font=small,capposition=top}

% 章的序号样式
\ctexset{
    chapter/number = \arabic{chapter},
    chapter/numberformat = \color{blue}\zihao{0}\itshape,
}
% 调整字间距,I don't know the effect,也许ctex已经这样设置了,需要调查?
%\renewcommand{\CJKglue}{\hskip 0.9bp plus 0.03\baselineskip minus 0.03\baselineskip}
\input{/home/subaochen/git/writing-common/translations.tex}
\input{/home/subaochen/git/writing-common/listing.tex}
